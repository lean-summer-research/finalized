\chapter{Green's Relations}

% Begin MyProject/Semigroup/Green/Defs.lean

\begin{definition}[Green's Preorder and Equivalence Relations]
\label{def:greens-relations}
In a semigroup $S$, Green's relations are a set of five equivalence relations that characterize the elements of $S$ in terms of the principal ideals they generate. These relations are fundamental to the study of semigroups.

First, we define four preorder relations: $\leq_{\mathcal{R}}$, $\leq_{\mathcal{L}}$, $\leq_{\mathcal{J}}$, and $\leq_{\mathcal{H}}$. Let $S^1$ be the monoid obtained by adjoining an identity element to $S$ if it does not already have one. For any two elements $x, y \in S$:
\begin{itemize}
    \item $x \leq_{\mathcal{R}} y$ if and only if the principal right ideal generated by $x$ is a subset of the principal right ideal generated by $y$ ($xS^1 \subseteq yS^1$). In Lean, we use the equivalent definition that there must exist some $z \in S^1$ such that $x = yz$.
    \item $x \leq_{\mathcal{L}} y$ if and only if the principal left ideal generated by $x$ is a subset of the principal left ideal generated by $y$ ($S^1x \subseteq S^1y$). In Lean, we use the equivalent definition that there must exist some $z \in S^1$ such that $x = zy$.
    \item $x \leq_{\mathcal{J}} y$ if and only if the principal two-sided ideal generated by $x$ is a subset of the principal two-sided ideal generated by $y$ ($S^1xS^1 \subseteq S^1yS^1$). In Lean, we use the equivalent definition that there must exist some $u, v \in S^1$ such that $x = uyv$.
    \item $x \leq_{\mathcal{H}} y$ if and only if $x \leq_{\mathcal{R}} y$ and $x \leq_{\mathcal{L}} y$.
\end{itemize}
These four relations are preorders (they are reflexive and transitive).

From these preorders, we define the equivalence relations $\mathcal{R}$, $\mathcal{L}$, $\mathcal{J}$, and $\mathcal{H}$ as the symmetric closures of their corresponding preorders. For example, $x \mathcal{R} y$ if and only if $x \leq_{\mathcal{R}} y$ and $y \leq_{\mathcal{R}} x$. The equivalence classes are denoted by $[x]_{\mathcal{R}}$, $[x]_{\mathcal{L}}$, etc.

Finally, the $\mathcal{D}$ relation is defined by composing $\mathcal{R}$ and $\mathcal{L}$: $x \mathcal{D} y$ if there exists an element $z \in S$ such that $x \mathcal{R} z$ and $z \mathcal{L} y$. It can be shown that $\mathcal{D}$ is an equivalence relation and that it can also be defined by composing $\mathcal{L}$ and $\mathcal{R}$.

\lean{Semigroup.RPreorder, Semigroup.LPreorder, Semigroup.JPreorder, Semigroup.HPreorder, Semigroup.RPreorder.isPreorder, Semigroup.LPreorder.isPreorder, Semigroup.JPreorder.isPreorder, Semigroup.HPreorder.isPreorder, Semigroup.REquiv, Semigroup.LEquiv, Semigroup.JEquiv, Semigroup.HEquiv, Semigroup.DEquiv, Semigroup.REquiv.isEquivalence, Semigroup.LEquiv.isEquivalence, Semigroup.JEquiv.isEquivalence, Semigroup.HEquiv.isEquivalence, Semigroup.DEquiv.isEquivalence}
\leanok
\end{definition}

\begin{lemma}[Multiplication Compatibility of Green's Relations]
\label{lem:greens-relations-mul-compat}
The $\mathcal{R}$ and $\mathcal{L}$ relations exhibit compatibility with semigroup multiplication on one side. Specifically, the $\mathcal{R}$-preorder is compatible with left multiplication, and the $\mathcal{L}$-preorder is compatible with right multiplication.
If $x \leq_{\mathcal{R}} y$, then for any $z \in S$, we have $zx \leq_{\mathcal{R}} zy$.
This property extends to the equivalence relation $\mathcal{R}$. If $x \mathcal{R} y$, then $zx \mathcal{R} zy$.
A similar argument holds for the $\mathcal{L}$-preorder and $\mathcal{L}$-equivalence, which are compatible with right multiplication. If $x \leq_{\mathcal{L}} y$, then $xz \leq_{\mathcal{L}} yz$ for any $z \in S$.
\lean{Semigroup.RPreorder.lmult_compat, Semigroup.REquiv.lmult_compat, Semigroup.LPreorder.rmult_compat, Semigroup.LEquiv.rmult_compat}
\leanok
\uses{def:greens-relations}
\end{lemma}

\begin{proof}
\leanok
Let $x, y, z \in S$.
To prove left compatibility for $\leq_{\mathcal{R}}$, assume $x \leq_{\mathcal{R}} y$. By definition, there exists $a \in S^1$ such that $x = ya$. Multiplying by $z$ on the left gives $zx = z(ya) = (zy)a$. This implies $zx \leq_{\mathcal{R}} zy$.
For the equivalence $x \mathcal{R} y$, we have both $x \leq_{\mathcal{R}} y$ and $y \leq_{\mathcal{R}} x$. Applying the result for preorders, we get $zx \leq_{\mathcal{R}} zy$ and $zy \leq_{\mathcal{R}} zx$, which means $zx \mathcal{R} zy$.
The proof for $\leq_{\mathcal{L}}$ and $\mathcal{L}$ with right multiplication is analogous.
\uses{def:greens-relations}
\end{proof}

\begin{lemma}[Commutation of R and L Relations]
\label{lem:r-l-comm}
The composition of the relations $\mathcal{R}$ and $\mathcal{L}$ is commutative. That is, for any $x, y \in S$, there exists a $z$ such that $x \mathcal{R} z$ and $z \mathcal{L} y$ if and only if there exists a $w$ such that $x \mathcal{L} w$ and $w \mathcal{R} y$. This can be written as $\mathcal{R} \circ \mathcal{L} = \mathcal{L} \circ \mathcal{R}$. This property is crucial for proving that the relation $\mathcal{D} = \mathcal{R} \circ \mathcal{L}$ is symmetric, and therefore an equivalence relation.
\lean{Semigroup.rEquiv_lEquiv_comm}
\leanok
\uses{def:greens-relations, lem:greens-relations-mul-compat}
\end{lemma}

\begin{proof}
\leanok
Suppose there exists $z$ with $x \mathcal{R} z$ and $z \mathcal{L} y$. From $x \mathcal{R} z$, we have $x=za$ and $z=xb$ for some $a,b \in S^1$. From $z \mathcal{L} y$, we have $z=cy$ and $y=dz$ for some $c,d \in S^1$. We need to find an element $w$ such that $x \mathcal{L} w$ and $w \mathcal{R} y$. The Lean proof shows that in the non-trivial case, the element $dza$ can be used for $w$. This commutation is essential for establishing that $\mathcal{D}$ is an equivalence relation, as it directly implies symmetry.
\uses{def:greens-relations}
\end{proof}

\begin{lemma}[Closure of D under R and L]
\label{lem:d-equiv-closed}
The $\mathcal{D}$ relation is closed under composition with $\mathcal{R}$ and $\mathcal{L}$. If $x \mathcal{D} y$ and $y \mathcal{L} z$, then $x \mathcal{D} z$. Similarly, if $x \mathcal{D} y$ and $y \mathcal{R} z$, then $x \mathcal{D} z$. This property is used to prove the transitivity of $\mathcal{D}$.
\lean{Semigroup.DEquiv.closed_under_lEquiv, Semigroup.DEquiv.closed_under_rEquiv}
\leanok
\uses{def:greens-relations}
\end{lemma}

\begin{proof}
\leanok
Suppose $x \mathcal{D} y$ and $y \mathcal{L} z$. By definition of $\mathcal{D}$, there exists an element $w$ such that $x \mathcal{R} w$ and $w \mathcal{L} y$. Since $\mathcal{L}$ is an equivalence relation, from $w \mathcal{L} y$ and $y \mathcal{L} z$, we can deduce $w \mathcal{L} z$. Now we have $x \mathcal{R} w$ and $w \mathcal{L} z$, which by definition means $x \mathcal{D} z$.
A similar argument holds for closure under $\mathcal{R}$. If $x \mathcal{D} y$ and $y \mathcal{R} z$, we use the commutation of $\mathcal{R}$ and $\mathcal{L}$ (\ref{lem:r-l-comm}). $x \mathcal{D} y$ means there is a $w$ with $x \mathcal{L} w$ and $w \mathcal{R} y$. Since $\mathcal{R}$ is an equivalence relation, $w \mathcal{R} y$ and $y \mathcal{R} z$ implies $w \mathcal{R} z$. So we have $x \mathcal{L} w$ and $w \mathcal{R} z$, which means $x \mathcal{D} z$.
\uses{def:greens-relations, lem:r-l-comm}
\end{proof}

% End MyProject/Semigroup/Green/Defs.lean
